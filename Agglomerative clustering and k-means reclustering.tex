\documentclass[a4paper]{article}
\usepackage[utf8]{inputenc}
\usepackage[T1]{fontenc}
\usepackage{titling}



%% Better math support:
%\usepackage{amsmath}

%% Bibliography style:
\usepackage{mathptmx}           % Use the Times font.
\usepackage{graphicx}           % Needed for including graphics.
\usepackage{url}                % Facility for activating URLs.

%% Set the paper size to be A4, with a 2cm margin 
%% all around the page.
\usepackage[a4paper,margin=2cm]{geometry}

%% Natbib is a popular style for formatting references.
\usepackage{natbib}
%% bibpunct sets the punctuation used for formatting citations.
\bibpunct{(}{)}{;}{a}{,}{,}

%% textcomp provides extra control sequences for accessing text symbols:
\usepackage{textcomp}
\newcommand*{\micro}{\textmu}
%% Here, we define the \micro command to print a text "mu".
%% "\newcommand" returns an error if "\micro" is already defined.

%% This is an example of a new macro that I've created to save me
%% having to type \LaTeX each time.  The xspace command provides space
%% after the word LaTeX where appropriate.
\usepackage{xspace}
\providecommand*{\latex}{\LaTeX\xspace}


\begin{document}
\title{Agglomerative clustering and k-means reclustering}
%\subtitle{A data mining project made in Warsaw University of Technology in the Mathematics and %Information  science faculty }
\author{(Leo Lautsch and Jean-Baptiste Gastineau)}
\maketitle



\begin{abstract}
  The purpose of this short document is to explain and discuss one implementation of the "Agglomerative and k-means clustering". This project is part of the lecture Data Mining from the Mathematics and Computer science faculty of Warsaw university of technology.
\end{abstract}


\newpage


\section{Introduction}
\section{Presentation of the data set}
\section{Implementation}
\subsection{Une sous-section}
Il était une fois dans une lointaine galaxie...
\subsection{Une deuxième sous-section}
La guerre des clones faisait alors rage...

\newpage

 \tableofcontents 
\end{document}
